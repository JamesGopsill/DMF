\documentclass[11pt]{article}
\usepackage[T1]{fontenc}
\usepackage[a4paper, left=2cm, right=2cm, top=2cm, bottom=2cm, footskip=0.2cm, headsep=0.2cm]{geometry}
\usepackage[english]{babel}
\usepackage{blindtext}
\usepackage{sectsty}
\usepackage{fancyhdr}
\usepackage{graphicx}
\usepackage{amsmath}
\usepackage{natbib}
\usepackage{lastpage}

\pagestyle{fancy}
\fancyhf{}
\lhead{Feasibility Design Report}
\rhead{Design \& Manufacture 2 Submission}
\rfoot{\hrule Page \thepage{} of \pageref{LastPage}}
\lfoot{}

\sectionfont{\sc}
\subsectionfont{\sc}
\subsubsectionfont{\sc}

%%%%%%%%%%%%%%%%%%% Your Content Below %%%%%%%%%%%%%%%%%%%%%%%%%%%%%

\title{Feasibility Design Report}
\author{
Joe Bloggs (XXXXX),
John Doe (XXXXX)
}

\begin{document}

\maketitle
\thispagestyle{fancy}

\begin{abstract}

  This is the abstract of the report and should contain a summary of the report. Abstracts are typically around two paragraphs in length and contain the background, contents of the report and some key findings.

\end{abstract}

\section{introduction}

The introduction of the report should provide some context surrounding the design exercise. You should do some research on the convertible car market, discuss some key features of a car convertible roof and describe the key factors that need to be considered. The introduction should also discuss the layout of the report and how the sections feed into one another.

\section{product design specification}

This section should provide details of how the PDS was generated as well as presenting the PDS for you car convertible roof. We expect to see some research into existing designs to help you define your specifications.

\section{concept generation}

This section should include a description of the process you used to generate your concepts and the key features that you will be investigating at this early stage. You will the describe each concept in detail and in relation to the features you have highlighted.

\subsection{concept one}

\subsection{concept two}

\subsection{concept three}

\section{concept selection}

Once you have generated your three concepts, you will report on the selection process in order to decide upon your the concept you will take forward to preliminary design.

\section{deployment modelling}

This section should present and discuss the deployment model you have made during this project. You should include your initial calculations and details of the simulink model. Remember to introduct each section so the reader has an idea of what they about to read.

\subsection{initial calculations}

\subsection{simulink model}

\subsubsection{assumptions}

\subsubsection{motor \& gearbox model}

\subsubsection{mechanism model}

\section{motor, gear ratio \& damping selection}

This section should use the model that has been described in the previous section and report on the investigation you have performed to investigate the sensitivity of your concpet to changes in motor, gear ratio \& damping parameters. You should then recommend the motor, gear ratio \& damping for your design.

\section{gearbox design}

This section should discuss the two types of gearbox that you will be considering in this project and report back a potential solution for each type. You will then review these concepts and selec one to carry forward as part of your solution.

\subsection{helical gearbox}

\subsection{spur gearbox}

\subsection{gearbox selection}

\section{solution specification}

This section should present the solution specification for your model and recommendations for the motor \& gearbox design.

\section{conclusion \& future work}

The conclusion should summarise the exercise and highlight the key findings. You should also discuss what would be the next steps in taking this design further.

\section*{references}

\end{document}
